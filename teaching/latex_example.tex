\documentclass{article}

% Language setting
% Replace `english' with e.g. `spanish' to change the document language
\usepackage[greek,english]{babel}  % For Greek and English languages
\usepackage[utf8]{inputenc}
%\usepackage[iso-8859-7]{inputenc}
\usepackage{amsmath}  % For math mode Greek letters
\usepackage{textgreek}  % For text mode Greek letters
\usepackage{alphabeta}  % For mixed Greek and Latin scripts

% Set page size and margins
% Replace `letterpaper' with `a4paper' for UK/EU standard size
\usepackage[letterpaper,top=2cm,bottom=2cm,left=3cm,right=3cm,marginparwidth=1.75cm]{geometry}

% Useful packages
\usepackage{amsfonts}
\usepackage{mathabx}
\usepackage{graphicx}
\usepackage[colorlinks=true, allcolors=blue]{hyperref}

\title{Γραμμική Άλγεβρα - Σετ Ασκήσεων 1}
%\author{Konstantinos Skianis}
\date{}

\begin{document}
\maketitle

% \begin{abstract}
% Your abstract.
% \end{abstract}

% \section*{Matrices}

\subsection*{Άσκηση 1}

Eπαληθεύστε την ιδιότητα \((AB)\Gamma = A(B\Gamma)\) για τους δοθέντες πίνακες: \\

\( A = \begin{bmatrix} 2 & 1 \\ 3 & 4 \end{bmatrix} \),
\( B = \begin{bmatrix} 0 & 3 & 7 \\ 1 & 8 & 9 \end{bmatrix} \),
\( \Gamma = \begin{bmatrix} 3 & 7 & 1 \\ 2 & 6 & 1 \\ 1 & 4 & 0 \end{bmatrix} \)

\subsection*{Άσκηση 2}
Για τον πίνακα:

\[
A = \begin{bmatrix} 1 & 2 \\ 3 & -4 \end{bmatrix}
\]

να βρεθεί πίνακας $X \in M_{2\times1}$ τέτοιος ώστε $ΑX=2X$.

\subsection*{Άσκηση 3}
Χαρακτηρίστε κάθε πρόταση ως Σωστή ή Λάθος, δικαιολογώντας την απάντηση σας.
\begin{enumerate}
\item Αν $Α$ είναι $3\times3$ πίνακας, τότε $det(7A)=7(detA)$
\item Αν οι Α και Β είναι $n\times n$ πίνακες με $detA=2$ και $detB=3$ τότε $det(A+B)=5$ και $det(A^3)=6$
\item $det(-A)=-det(A)$
\end{enumerate}

\subsection*{Άσκηση 4}
Να λυθεί το σύστημα χρησιμοποιώντας τη μέθοδο του Gauss:

\[
\begin{cases}
x_1 + x_2 - 2x_3 + x_4 + 3x_5 = 1 \\
2x_1 - x_2 + 2x_3 + 2x_4 + 6x_5 = 2 \\
3x_1 + 2x_2 - 4x_3 - 3x_4 - 9x_5 = 3 \\
\end{cases}
\]

\subsection*{Άσκηση 5}
Να υπολογιστεί η ορίζουσα:

\[
\begin{vmatrix}
2 & -3 & 2 & 5 \\
1 & -1 & 1 & 2 \\
3 & 2 & 2 & 1 \\
1 & 1 & -3 & -1 \\
\end{vmatrix}
\]


\subsection*{Άσκηση 6}
Να λυθεί το σύστημα (να χρησιμοποιηθεί και η μέθοδος Cramer):

\[
\begin{cases}
x_1 + x_2 - x_3 = 1 \\
x_1 + \alpha x_2 + 3x_3 = 2 \\
2x_1 + 3x_2 + \alpha x_3 = 3 \\
\end{cases}
\]

\subsection*{Latex παράδειγμα (Gauss)}

1ος τρόπος:
\[
\left(
\begin{array}{rrr}
-2 & 8  & 0 \\
\textcolor{red}{10} & \textcolor{red}{0}  & \textcolor{red}{-1}\\
\textcolor{blue}{0} & \textcolor{blue}{-3}  & \textcolor{blue}{4}\\
\end{array}
\right)
\xrightarrow{r_2 \leftrightarrow r_3}
\left(
\begin{array}{rrr}
-2 & 8  & 0 \\
\textcolor{blue}{0} & \textcolor{blue}{-3}  & \textcolor{blue}{4}\\
\textcolor{red}{10} & \textcolor{red}{0}  & \textcolor{red}{-1}\\
\end{array}
\right)
\]

\noindent 2ος τρόπος:

%\begin{scriptsize}
\[
r_1 \rightarrow r_1 + 3 \, r_2: \quad
\left(
\begin{array}{rrr}
\textcolor{red}{-2} & \textcolor{red}{8}  & \textcolor{red}{0} \\
\textcolor{blue}{10} & \textcolor{blue}{0}  & \textcolor{blue}{-1}\\
0  & -3 & 4 \\
\end{array}
\right)
\begin{array}{c}
\Lsh \\
(3)\\
 \\
\end{array}
\sim
\left(
\begin{array}{rrr}
\textcolor{red}{28} & \textcolor{red}{8}  & \textcolor{red}{-3} \\
\textcolor{blue}{10} & \textcolor{blue}{0}  & \textcolor{blue}{-1}\\
0  & -3 & 4 \\
\end{array}
\right)
\]

\vspace{1cm}
\[
r_3 \rightarrow r_3 - r_2: \quad
\left(
\begin{array}{rrr}
-2 & 8 & 0\\
\textcolor{blue}{10} & \textcolor{blue}{0}  & \textcolor{blue}{-1}\\
\textcolor{red}{0} & \textcolor{red}{-3}  & \textcolor{red}{4} \\
\end{array}
\right)
\begin{array}{c}
 \\
(-1)\\
\mathbin{\rotatebox[origin=c]{180}{$\Rsh$}}\\
\end{array}
\sim
\left(
\begin{array}{rrr}
-2 & 8 & 0\\
\textcolor{blue}{10} & \textcolor{blue}{0}  & \textcolor{blue}{-1}\\
\textcolor{red}{-10} & \textcolor{red}{-3}  & \textcolor{red}{5} \\
\end{array}
\right)
\]


\end{document}

% other links: https://www.overleaf.com/learn/latex/List_of_Greek_letters_and_math_symbols 
